%% Adaptado de 
%% http://www.ctan.org/tex-archive/macros/latex/contrib/IEEEtran/
%% Traduzido para o congresso de IC da USP
%%*****************************************************************************
% Não modificar

\documentclass[twoside,conference,a4paper]{IEEEtran}

%******************************************************************************
% Não modificar
\usepackage{IEEEtsup} % Definições complementares e modificações.

\usepackage[english,brazil]{babel}
\usepackage[utf8]{inputenc}

%% \usepackage{latexsym,amsfonts} % Disponibiliza fontes adicionais.

\usepackage[T1]{fontenc}
\usepackage{fbb}

\usepackage{animate}

\usepackage{theorem} 
%% \usepackage[cmex10]{amsmath} % Pacote matemático básico 
\usepackage{url} 
\usepackage{graphicx}
\usepackage{amsmath}
\usepackage{amssymb}
\usepackage{color}
\usepackage[pagebackref=true,breaklinks=true,colorlinks,bookmarks=false]{hyperref}
\usepackage[tight,footnotesize]{subfigure} 
\usepackage[noadjust]{cite} % Disponibiliza melhorias em citações.
%%*****************************************************************************

\begin{document}
\selectlanguage{brazil}
\renewcommand{\IEEEkeywordsname}{Palavras-chave}

%%*****************************************************************************

\urlstyle{tt}
% Indicar o nome do autor e o curso/nível (grad-mestrado-doutorado-especial)
\title{Título do Trabalho}
\author{%
 \IEEEauthorblockN{Lucy Miyuki Miyagusiku Narita\,\IEEEauthorrefmark{1}}
 \IEEEauthorblockA{\IEEEauthorrefmark{1}%
                   Ciência da Computação - Graduação \\
                   E-mail: ra182851@students.ic.unicamp.br}
}

%%*****************************************************************************

\maketitle

%%*****************************************************************************
% Resumo do trabalho
\begin{abstract}
 O resumo deve conter uma breve descrição sobre várias partes do seu trabalho que serão tratadas no decorrer do texto. Primeiramente, pode-se descrever brevemente o problema no qual você está trabalhando: Por que você está desenvolvendo este trabalho? Qual a motivação para este desenvolvimento? Por que ele é importante? O resumo deve conter também um breve descritivo da metodologia que você usou no desenvolvimento: Que problema foi tratado? Como a solução foi construída/desenvolvida? Quais as tecnologias utilizadas? Finalmente, deve falar um pouco sobre os resultados que você conseguiu: o resultado final ficou bom? Quais os seus principais diferenciais? Qual a eficiência do desenvolvimento?
\end{abstract}

% Indique três palavras-chave que descrevem o trabalho
\begin{IEEEkeywords}
 Palavras-chave
\end{IEEEkeywords}

%%*****************************************************************************
% Modifique as seções de acordo com o seu projeto

\section{Introdução}

Na introdução você deve descrever os aspectos mais relevantes sobre a revisão bibliográfica que fez e do problema que você decidiu tratar. Quais foram os pontos estudados/pesquisados? Quais os outros trabalhos similares ao seu que você encontrou? 

Também na introdução espera-se que você descreva um pouco sobre a motivação de trabalhar com esse tema. A descrição do seu trabalho será feita em detalhes nas próximas seções do artigo.


No final da introdução, é comum inserir um parágrafo descrevendo o que será encontrado em cada seção no restante do seu texto. Exemplo: Este trabalho encontra-se organizado da seguinte forma: a seção 2 apresenta X. A seção 3 descreve Y. Os resultados são apresentados na seção 4, e as conclusões são apresentadas na seção 5.

\section{Dependências}

O projeto foi desenvolvido em Python 3.X. As dependências se encontram no arquivo \texttt{requirements.txt}.

\section{Trabalho Proposto}

Nesta seção descreva de forma abrangente, porém clara e organizada, o seu trabalho.

\subsection{O Modelo}

Descrição do indivíduo e da população.

\section{Resultados e Discussão}

Nesta seção você deve apresentar claramente os resultados obtidos para os testes efetuados. Procure organizar os dados utilizando uma linguagem científica. Algumas opções são o uso de tabelas e gráficos, para que a compreensão seja fácil e rápida.

\begin{itemize}
    \item tamanho da população
    \item critério de parada
    \item técnica de seleção
    \item técnica de crossover
    \item técnica de mutação
    \item método de substituição
    \item taxa de mutação
    \item taxa de crossover
\end{itemize}

\section{Conclusões}

Nesta seção, faça uma análise geral de seu trabalho, levando em conta todo o processo de desenvolvimento e os resultados. Quais os seus pontos fortes? Quais os seus pontos fracos? Quais aspectos de sua metodologia de trabalho foram positivas? Quais foram negativas? O que você recomendaria (ou não recomendaria) a outras pessoas que estejam realizando trabalhos similares aos seus? 


%******************************************************************************
% Referências - Definidas no arquivo Relatorio.bib
 +-------------+

\bibliographystyle{IEEEtran}

\bibliography{Relatorio}


%******************************************************************************


\end{document}

%% Adaptado de 
%% http://www.ctan.org/tex-archive/macros/latex/contrib/IEEEtran/
%% Traduzido para o congresso de IC da USP
%%*****************************************************************************
% Não modificar

\documentclass[twoside,conference,a4paper]{IEEEtran}

%******************************************************************************
% Não modificar
\usepackage{IEEEtsup} % Definições complementares e modificações.

\usepackage[english,brazil]{babel}
\usepackage[utf8]{inputenc}

%% \usepackage{latexsym,amsfonts} % Disponibiliza fontes adicionais.

\usepackage[T1]{fontenc}
\usepackage{fbb}

\usepackage{animate}

\usepackage{theorem} 
%% \usepackage[cmex10]{amsmath} % Pacote matemático básico 
\usepackage{url} 
\usepackage{graphicx}
\usepackage{amsmath}
\usepackage{amssymb}
\usepackage{color}
\usepackage[pagebackref=true,breaklinks=true,colorlinks,bookmarks=false]{hyperref}
\usepackage[tight,footnotesize]{subfigure} 
\usepackage[noadjust]{cite} % Disponibiliza melhorias em citações.
%%*****************************************************************************

\begin{document}
\selectlanguage{brazil}
\renewcommand{\IEEEkeywordsname}{Palavras-chave}

%%*****************************************************************************

\urlstyle{tt}
% Indicar o nome do autor e o curso/nível (grad-mestrado-doutorado-especial)
\title{Título do Trabalho}
\author{%
 \IEEEauthorblockN{João Vitor Araki Gonçalves\,\IEEEauthorrefmark{1}}
 \IEEEauthorblockN{Lucy Miyuki Miyagusiku Narita\,\IEEEauthorrefmark{2}}
 \IEEEauthorblockA{Ciência da Computação - Graduação \\
                   \IEEEauthorrefmark{1}%
                   ra176353@students.ic.unicamp.br\\
                   \IEEEauthorrefmark{2}%
                   ra182851@students.ic.unicamp.br}
}

%%*****************************************************************************

\maketitle

%%*****************************************************************************
% Modifique as seções de acordo com o seu projeto

\section{Introdução}

Este trabalho tem como objetivo implementar e aplicar uma técnica de computação evolutiva na prática.
Algoritmos genéticos problemas são resolvidos através de um processo evolutivo que resulta na solução mais adequada dado algum critério, são, portanto, muito utilizados para problemas de otimização e de busca onde o crescimento exponencial das combinações faz de um algoritmo tradicional de busca inviável.

\section{Dependências}

O projeto foi desenvolvido em Python 3.X. As dependências se encontram no arquivo \texttt{requirements.txt}.

\section{Trabalho Proposto}

As redes neurais são o ponto central da revolução do \emph{Deep Learning}. A tecnologia avançou a ponto de ser capaz de sintetizar voz, reconhecer imagens e alterar o conteúdo de videos de forma a adulterar a informação passada originalmente.

Para este projeto, procuramos fazer o \emph{tunning} dos parâmetros de uma rede neural com um algoritmo genético. Trabalho que normalmente é feito "à mão" com base em experimentação e determinação destes valores de forma empírica.

\subsection{Indivíduo (Cromossomo)}

Nosso indivíduo é composto por 4 valores reais:

\begin{itemize}
    \item número de epochs
    \item quantidade de layers
    \item output space dimension para cada layer
    \item número de features
\end{itemize}

\section{Resultados e Discussão}

Nesta seção você deve apresentar claramente os resultados obtidos para os testes efetuados. Procure organizar os dados utilizando uma linguagem científica. Algumas opções são o uso de tabelas e gráficos, para que a compreensão seja fácil e rápida.

\begin{itemize}
    \item tamanho da população
    \item critério de parada
    \item técnica de seleção
    \item técnica de crossover
    \item técnica de mutação
    \item método de substituição
    \item taxa de mutação
    \item taxa de crossover
\end{itemize}

\section{Conclusões}

Nesta seção, faça uma análise geral de seu trabalho, levando em conta todo o processo de desenvolvimento e os resultados. Quais os seus pontos fortes? Quais os seus pontos fracos? Quais aspectos de sua metodologia de trabalho foram positivas? Quais foram negativas? O que você recomendaria (ou não recomendaria) a outras pessoas que estejam realizando trabalhos similares aos seus? \cite{Rowling:1997}


%******************************************************************************
% Referências - Definidas no arquivo Relatorio.bib
 +----------------+

\bibliographystyle{IEEEtran}

\bibliography{Relatorio}


%******************************************************************************


\end{document}
